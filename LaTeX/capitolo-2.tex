\chapter{Model Context Protocol}

\section{Funzionamento}

MCP è un protocollo open-source che standardizza il modo in cui le applicazioni forniscono contesto agli LLM. Per analogia, si può pensare a MCP 
come a una porta USB-C per le applicazioni IA. Proprio come USB-C fornisce un modo standardizzato per collegare i dispositivi a varie periferiche 
e accessori, MCP fornisce un modo standardizzato per collegare i modelli IA a diverse fonti di dati e strumenti. MCP consente di creare agenti e 
flussi di lavoro complessi basati sugli LLM e connette i modelli con il mondo. 
\cite{modelcontextprotocol2024intro}

Il protocollo MCP include i seguenti progetti:
\begin{itemize}
\item Specifiche MCP: una specifica di MCP che delinea i requisiti di implementazione per client e server.
\item SDK MCP: \textit{Software Development Kit} (SDK) per diversi linguaggi di programmazione che implementano MCP.
\item Strumenti di sviluppo MCP.
\item Implementazioni di server MCP di riferimento.
\end{itemize}
MCP si concentra esclusivamente sul protocollo per lo scambio di contesto, senza stabilire come le applicazioni IA utilizzino gli LLM o gestiscano il contesto fornito.
\cite{modelcontextprotocol2024arch}

\subsection{Partecipanti}
MCP segue un'architettura client-server in cui un host MCP, un'applicazione IA come ad esempio Claude Desktop, stabilisce connessioni a uno o più server MCP. L'host MCP realizza questo creando un client MCP per ogni server MCP. Ogni client MCP mantiene una connessione uno-a-uno dedicata con il suo server MCP corrispondente.
I principali partecipanti all'architettura MCP sono:
\begin{itemize}
\item Host MCP: l'applicazione IA che coordina e gestisce uno o più client MCP.
\item Client MCP: un componente che mantiene una connessione a un server MCP e ottiene, da un server MCP, il contesto che l'host MCP può utilizzare.
\item Server MCP: un programma che fornisce contesto ai client MCP; possono essere eseguiti sia in locale che in remoto.
\cite{modelcontextprotocol2024arch}
\end{itemize}

\subsection{Livelli}
MCP è costituito da due livelli:
\begin{itemize}
\item Livello dati: definisce il protocollo, basato su JSON-RPC, per la comunicazione client-server, inclusa la gestione del ciclo di vita e le primitive principali, come strumenti, risorse, prompt e notifiche.
\item Livello trasporto: definisce i meccanismi e i canali di comunicazione che consentono lo scambio di dati tra client e server, inclusi l'instaurazione di connessioni specifiche per il trasporto, il framing dei messaggi e l'autorizzazione.
\end{itemize}
Concettualmente, il livello dati è il livello interno, mentre il livello trasporto è il livello esterno.
\cite{modelcontextprotocol2024arch}

\subsubsection{Livello dati}
Il livello dati implementa un protocollo di scambio basato su JSON-RPC 2.0 che definisce la struttura e la semantica dei messaggi. Questo livello include:
\begin{itemize}
\item Gestione del ciclo di vita: gestisce l'inizializzazione della connessione, la negoziazione delle capacità e la terminazione della connessione tra client e server.
\item Funzionalità del server: consente ai server di fornire funzionalità di base, inclusi strumenti per azioni IA, risorse per dati di contesto e richieste per schemi di interazione da e verso il client.
\item Funzionalità del client: consente ai server di chiedere al client di campionare dall'LLM, ottenere input dall'utente e registrare messaggi al client.
\item Funzionalità di utilità: supporta funzionalità aggiuntive come notifiche per aggiornamenti in tempo reale e monitoraggio dei progressi per operazioni di lunga durata.
\cite{modelcontextprotocol2024arch}
\end{itemize}

\subsubsection{Livello di trasporto}
Il livello di trasporto gestisce i canali di comunicazione e l'autenticazione tra client e server. Si occupa della creazione della connessione, il framing dei messaggi e la comunicazione sicura tra i partecipanti.
MCP supporta due meccanismi di trasporto:
\begin{itemize}
\item \textit{Stdio Transport}: utilizza flussi di input/output standard per la comunicazione diretta tra processi locali sulla stessa macchina, garantendo prestazioni ottimali senza sovraccarico di rete.
\item \textit{Streamable HTTP transport}: utilizza metodi HTTP POST per i messaggi client-server con eventi inviati dal server opzionali per le funzionalità di streaming. Questo trasporto consente la comunicazione con il server remoto e supporta metodi di autenticazione HTTP standard, inclusi \textit{bearer token}, chiavi API e intestazioni personalizzate.
\end{itemize}
MCP consiglia di utilizzare OAuth per ottenere i token di autenticazione.
Il livello di trasporto astrae i dettagli di comunicazione dal livello di protocollo, consentendo lo stesso formato di messaggio del protocollo JSON-RPC 2.0 su tutti i meccanismi di trasporto.
\cite{modelcontextprotocol2024arch}

\subsection{Protocollo del livello dati}
Una parte fondamentale di MCP è la definizione dello schema e della semantica tra client e server MCP. Il livello dati è la parte di MCP che definisce le modalità con cui gli sviluppatori possono condividere il contesto dai server MCP ai client MCP.
MCP utilizza JSON-RPC 2.0 come protocollo \textit{Remote Procedure Call} (RPC). Client e server si inviano richieste e rispondono di conseguenza. Le notifiche possono essere utilizzate quando non è richiesta alcuna risposta.
\cite{modelcontextprotocol2024arch}

\subsubsection{Primitive}
Le primitive MCP sono il concetto più importante all'interno di MCP. Definiscono ciò che client e server possono offrirsi reciprocamente. Queste primitive specificano i tipi di informazioni contestuali che possono essere condivise con le applicazioni IA e la gamma di azioni che possono essere eseguite.
MCP definisce tre primitive principali che i server possono esporre:
\begin{itemize}
\item \textit{Tools}: funzioni eseguibili che le applicazioni IA possono invocare per eseguire azioni (e.g. operazioni su file, chiamate API, query di database).
\item Risorse: fonti di dati che forniscono informazioni contestuali alle applicazioni IA (e.g. contenuto di file, record di database, risposte API).
\item Prompt: schemi riutilizzabili che aiutano a strutturare le interazioni con i modelli linguistici (e.g. prompt di sistema, prompt \textit{few-shot}).
\end{itemize}
Ogni tipo di primitiva ha metodi associati per la scoperta (\texttt{*/list}), il recupero (\texttt{*/get}) e, in alcuni casi, l'esecuzione (\texttt{tools/call}). I client MCP utilizzeranno i metodi \texttt{*/list} per scoprire le primitive disponibili. Ad esempio, un client può prima elencare tutti gli strumenti disponibili (\texttt{tools/list}) e poi eseguirli. Questa progettazione consente di creare elenchi dinamici. \\
Come esempio concreto, si consideri un server MCP che fornisce contesto su un database. Può esporre strumenti per interrogare il database, una risorsa che contiene lo schema del database e un prompt che include esempi di interazione con gli strumenti. \\
MCP definisce anche le primitive che i client possono esporre. Queste primitive consentono agli autori del server MCP di creare interazioni più ricche.
\begin{itemize}
\item Campionamento: consente ai server di richiedere il completamento del modello linguistico dall'applicazione IA del client. Questa funzionalità è utile quando gli autori del server desiderano accedere a un modello linguistico, ma vogliono rimanere indipendenti dal modello e non includere un SDK del modello linguistico nel proprio server MCP. Possono utilizzare il metodo \texttt{sampling/complete} per richiedere il completamento del modello linguistico dall'applicazione IA del client.
\item Elicitazione: consente ai server di richiedere informazioni aggiuntive agli utenti. Questa funzionalità è utile quando gli autori del server desiderano ottenere maggiori informazioni dall'utente o chiedere la conferma di un'azione. Possono utilizzare il metodo \texttt{elicitation/request} per richiedere informazioni aggiuntive all'utente.
\item Logging: consente ai server di inviare messaggi di log ai client a scopo di debug e monitoraggio.
\cite{modelcontextprotocol2024arch}
\end{itemize}

\subsubsection{Notifiche}
Il protocollo supporta notifiche in tempo reale per abilitare aggiornamenti dinamici tra server e client. Ad esempio, quando gli strumenti disponibili su un server cambiano, come quando vengono rese disponibili nuove funzionalità o vengono modificati strumenti esistenti, il server può inviare notifiche di aggiornamento per informare i client connessi di tali modifiche. Le notifiche vengono inviate come messaggi di notifica JSON-RPC 2.0 (senza attendere una risposta) e consentono ai server MCP di fornire aggiornamenti in tempo reale ai client connessi.
\cite{modelcontextprotocol2024arch}
