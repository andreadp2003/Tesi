\chapter{Large Language Model - Cosa sono e come funzionano}

\section{Breve storia}

All'inizio degli anni '90, i modelli statistici di IBM hanno aperto la strada alle tecniche di allineamento delle parole per la traduzione automatica. Durante gli anni 2000, con l'aumento dell'accesso diffuso a Internet, i ricercatori hanno iniziato a compilare enormi set di dati testuali dal web ("web as corpus" \cite{kilgarriff2003webascorpus}) per addestrare questi modelli linguistici statistici \cite{banko2001scaling}\cite{resnik2003webparallel}, chiamati modelli n-gram \cite{goodman2001progress}. \\
Andando oltre i modelli n-gram, nel 2000 i ricercatori hanno iniziato a utilizzare le reti neurali per addestrare i modelli linguistici \cite{xu2000annlm}. In seguito al successo delle reti neurali profonde ("Deep Neural Networks") nella classificazione delle immagini intorno al 2012 \cite{chen2021cnnreview}, architetture simili sono state adattate per compiti linguistici. Questo cambiamento è stato segnato dallo sviluppo di "word embeddings" (ad esempio, Word2Vec di Mikolov nel 2013) e modelli "seq2seq". Nel 2016, Google ha convertito il suo servizio di traduzione alla traduzione automatica neurale ("Neural Machine Translation"), sostituendo i modelli statistici basati su frasi con reti neurali profonde ricorrenti ("Deep Recurrent Neural Networks"). \\
Alla conferenza NeurIPS del 2017, i ricercatori di Google hanno introdotto l'architettura del trasformatore in un articolo, divenuto fondamentale nel settore, chiamato "Attention Is All You Need" \cite{vaswani2017attention}. L'obiettivo di questo articolo era quello di migliorare la tecnologia seq2seq del 2014 basandosi su un meccanismo chiamato attenzione, sviluppato nel 2014 \cite{bahdanau2014nmt}. L'anno successivo, nel 2018, è stato introdotto BERT, che è rapidamente diventato "onnipresente" \cite{rogers2020bertology}. Sebbene il trasformatore originale abbia sia blocchi encoder che decoder, BERT è un modello solo encoder. L'uso accademico e di ricerca di BERT ha iniziato a diminuire nel 2023, a seguito di rapidi miglioramenti nelle capacità dei modelli solo decoder (come GPT) di risolvere compiti tramite prompt \cite{movva2024topics}. \\
Sebbene GPT-1 ("Generative Pre-Trained Transformer"), modello solo decoder, sia stato introdotto nel 2018, è stato GPT-2 nel 2019 ad attirare l'attenzione generale, poichè OpenAI ha affermato di averlo inizialmente ritenuto troppo potente per essere rilasciato pubblicamente, per timore di un utilizzo dannoso \cite{hern2019fakeai}. Ma è stata l'applicazione ChatGPT del 2022, con la sua interfaccia utente che riprende il concetto di chatbot, a ricevere un'ampia copertura mediatica e l'attenzione del pubblico \cite{euronews2023chatgpt}. GPT-4 del 2023 è stato elogiato per la sua maggiore accuratezza e definito un "Sacro Graal" per le sue capacità multimodali (i.e.: la sua capacità di analizzare dati di diverso tipo, come testo, audio e immagini) \cite{heaven2023gpt4}. Il rilascio di ChatGPT ha portato ad un aumento nell'utilizzo degli LLM in diversi sottocampi di ricerca dell'informatica, tra cui robotica ed ingegneria del software, oltre che ad un maggior riguardo per il loro impatto sociale \cite{movva2024topics}. Nel 2024 OpenAI ha rilasciato il modello di ragionamento OpenAI o1, che genera lunghe catene di pensiero prima di restituire una risposta finale \cite{metz2024openai}. Negli anni sono stati sviluppati molti LLM con dimensioni paragonabili a quelli della serie GPT di OpenAI \cite{ourworldindata2023parameters}. \\
Dal 2022, i modelli linguistici open-source hanno guadagnato popolarità, soprattutto con i primi modelli BLOOM e LLaMA di Meta e anche i modelli Mistral 7B e Mixtral 8x7b di Mistral AI. Nel gennaio 2025, DeepSeek ha rilasciato DeepSeek R1, un modello open-weight da 671 miliardi di parametri che offre prestazioni paragonabili a OpenAI o1, ma a un costo molto inferiore \cite{sharma2025deepseek}. \\
Dal 2023, molti LLM sono stati addestrati per essere multimodali, con la capacità di elaborare o generare anche altri tipi di dati, come immagini o audio. Questi LLM sono anche chiamati grandi modelli multimodali ("Large Multimodal Models") \cite{zia2024multimodal}. \\
Ad oggi, i modelli più grandi e performanti sono tutti basati sull'architettura a trasformatore \cite{merritt2022transformer}.

\section{Accenni del funzionamento}

Fondamentalmente, un Large Language Model (LLM) è in grado di prevedere quale sia la parola più plausibile da inserire alla fine di un testo dato, un processo che in gergo tecnico si chiama inferenza. \cite{LLM_next_token_prediction} \\
Poiché gli algoritmi di Machine Learning possono elaborare esclusivamente informazioni numeriche, il testo in ingresso viene innanzitutto suddiviso in frammenti più piccoli, chiamati \textit{token}. Un token non coincide necessariamente con una parola intera: può rappresentare una parola completa, una parte di essa o anche un singolo carattere, a seconda della suddivisione stabilita dal sistema di tokenizzazione \cite{tokens_tokenization}. \\
Ogni token viene poi convertito in un vettore numerico, detto \textit{embedding}. Ogni embedding rappresenta il token in uno spazio a molte dimensioni (spesso centinaia o migliaia), in cui la distanza e la direzione tra vettori riflettono relazioni semantiche e sintattiche tra parole. Ad esempio, gli embedding di “Italia” e “Francia” saranno vicini in quanto entrambe le parole rappresentano nazioni europee. In questo spazio vettoriale si trovano tutti gli embedding dei token che il modello ha incontrato durante la fase di addestramento. È proprio in questa fase che l'algoritmo “impara” a trasformare i token in embedding significativi, modificando gradualmente i propri parametri — talvolta dell'ordine di miliardi — in modo da ridurre l'errore nelle previsioni e aumentare la coerenza dei testi generati \cite{LLM_fine_tuning}. \\
Come accennato prima, i moderni LLM si basano sull'architettura \textit{transformer}, caratterizzata dall'uso del meccanismo dell'attenzione (\textit{attention mechanism}). All'interno di questa architettura, i vari embedding vengono manipolati matematicamente attraverso una serie di passaggi che includono moltiplicazioni di matrici, normalizzazioni e funzioni di attivazione, fino a produrre un vettore finale corrispondente alla previsione di un token specifico. Una volta generato questo token, viene aggiunto alla fine del testo e l'intero contenuto aggiornato viene nuovamente elaborato dal modello, ripetendo il ciclo fino a costruire un testo completo in risposta al prompt dell'utente \cite{transformer_self_attention}. \\
Il meccanismo dell'attenzione è un passaggio cruciale: esso permette al modello di valutare, per ogni token in ingresso, quali altri token della sequenza siano più rilevanti per determinarne il significato. In altre parole, non si limita a interpretare un token in maniera isolata, ma lo considera nel contesto più ampio della frase o del paragrafo in cui si trova. Il termine “attenzione” è ispirato a un processo analogo nella cognizione umana, in cui focalizziamo la nostra concentrazione su determinate parti di un testo per comprenderne appieno il senso. Grazie a questa capacità, gli LLM riescono a mantenere coerenza logica anche su sequenze testuali lunghe, collegando correttamente informazioni distanti tra loro e producendo risposte che seguono un filo narrativo o argomentativo esteso \cite{attention_human_cognition}. \\
L'introduzione di questo meccanismo ha segnato un punto di svolta nel settore. Tuttavia, la sua applicazione su larga scala è stata possibile solo grazie ai progressi nell'hardware, in particolare all'uso delle GPU (\textit{Graphics Processing Unit}). Le GPU, inizialmente progettate per l'elaborazione grafica, sono straordinariamente efficienti nel calcolo parallelo di operazioni matriciali, proprio quelle su cui si basano il meccanismo dell'attenzione e l'intera architettura transformer \cite{transformer_parallelizable}. Questo connubio tra innovazione algoritmica e potenza di calcolo ha permesso di addestrare modelli con miliardi di parametri su dataset enormi, aprendo la strada alla generazione di testi complessi e coerenti su scala mai vista prima. \\
Durante l'addestramento iniziale, il modello viene esposto ad enormi quantità di testo provenienti da fonti eterogenee, imparando a prevedere il token successivo dato un contesto. Per ogni previsione calcola l'errore rispetto al token reale e regola i propri miliardi di parametri tramite il processo di \textit{backpropagation}, ottimizzandoli gradualmente per ridurre l'errore medio. \\
Un modello già addestrato può essere ulteriormente specializzato tramite fine-tuning, che consiste nel riaddestrarlo (in parte o per intero) su un set di dati più ristretto e specifico, ad esempio testi tecnici o conversazioni in uno stile particolare. Questa procedura permette di adattare l'LLM a compiti specifici — come assistenza clienti, generazione di codice o analisi di documenti giuridici — senza ripetere da zero il costoso addestramento iniziale \cite{fine_tuning_transfer_learning}. \\
Un'ulteriore tecnica molto diffusa è RLHF (\textit{Reinforcement Learning from Human Feedback}), utilizzata per rendere le risposte del modello più utili, sicure e in linea con i valori umani. In questo approccio, dopo l'addestramento iniziale, il modello genera diverse possibili risposte ad un insieme di prompt; queste risposte vengono valutate da valutatori umani, che le classificano in base a criteri come accuratezza, cortesia o pertinenza. Con queste valutazioni, si addestra un modello a parte capace di sostituire, il più accuratamente possibile, il compito dei valutatori umani. L'LLM viene così ottimizzato per soddisfare al meglio la valutazione di questo modello \cite{RLHF_reward_model_from_humans}.

