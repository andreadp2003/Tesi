\setcounter{secnumdepth}{-1}

Il \textit{Ticket Management System} è pensato per essere posizionato all'interno di un contesto più grande, ossia all'interno di un sistema software
per il quale un utente può avere necessità di un supporto tecnico. Il sistema che verrà preso in esame è 3FS{\textsuperscript{\tiny\textcopyright}}.

\section{\texorpdfstring{3FS{\textsuperscript{\tiny\textcopyright}} - Panoramica del funzionamento}{3FS (c) - Panoramica del funzionamento}}

3FS{\textsuperscript{\tiny\textcopyright}} (\textit{Fleet and Financial Flexible System}) è una piattaforma avanzata per la gestione delle flotte aziendali, 
che automatizza processi operativi, commerciali, contabili e normativi. Ottimizza il ciclo di vita dei veicoli gestendo accordi, rotazione, acquisti, configurazione 
della flotta e distribuzione, con monitoraggio logistico e riduzione dei rischi di non conformità.

Il sistema automatizza adempimenti come registrazione targhe, tasse, assicurazioni e rende i veicoli rapidamente disponibili al noleggio. Gestisce manutenzione, rinnovi, 
tasse, sinistri e interagisce con le autorità per la gestione delle multe. Supporta la dismissione dei veicoli con processi automatizzati e integrazione contabile.

Sul piano finanziario, consente la registrazione e riconciliazione in tempo reale di fatture, gestione dei flussi di cassa e reportistica, integrandosi con sistemi 
contabili esterni per maggiore controllo e trasparenza. \cite{progesoftware_3fs}

\newpage
\section{\texorpdfstring{Ticket Management System applicato a 3FS{\textsuperscript{\tiny\textcopyright}}}{Il Ticket Management System applicato a 3FS (c)}}
Il Ticket Management System è per ora solo un proof-of-concept ma, in futuro, si potrà integrare all'interno di 3FS{\textsuperscript{\tiny\textcopyright}} per fornire supporto tecnico agli utenti.
I ticket saranno dunque inerenti a problemi legati alla piattaforma in questione, il che presuppone che l'LLM abbia una conoscenza dettagliata del suo funzionamento,
anche per poter rispondere in modo autonomo a problematiche semplici.

\subsection{Vantaggi attesi}
L'integrazione del Ticket Management System con 3FS{\textsuperscript{\tiny\textcopyright}} mira a ottenere diversi benefici:
\begin{itemize}
    \item riduzione del tempo medio di risoluzione dei problemi;
    \item diminuzione del carico di lavoro che grava sugli operatori umani;
    \item miglioramento della qualità del servizio e della soddisfazione dell'utente;
\end{itemize}

\newpage
\section{Conclusione}
Il Model Context Protocol offre un nuovo modo per potenziare le capacità degli LLM, già di per sé strumenti estremamente capaci e potenti.
Questa tesi ha illustrato una possibile applicazione di questa tecnologia.
L'uso di questo protocollo può permettere nuove forme di interazione tra modelli linguistici e sistemi aziendali 
complessi, aprendo la strada a un'integrazione sempre più forte tra intelligenza artificiale e processi produttivi.