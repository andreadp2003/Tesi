Il \textit{Ticket Management System} è pensato per essere posizionato all'interno di un contesto più grande, ossia all'interno di un sistema software
per il quale un utente può avere necessità di un supporto tecnico. Il sistema che verrà preso in esame è 3FS{\textsuperscript{\tiny\textcopyright}}.

\section{3FS{\textsuperscript{\tiny\textcopyright}} - Panoramica del funzionamento}
Il sistema 3FS{\textsuperscript{\tiny\textcopyright}} (\textit{Fleet and Financial Flexible System}) rappresenta una soluzione software avanzata per la gestione integrata delle flotte aziendali, 
concepita per rispondere in maniera automatizzata e completa ad un'ampia gamma di esigenze operative, commerciali, contabili, finanziarie e normative.

L'obiettivo primario della piattaforma è quello di ottimizzare l'intero ciclo di vita dei veicoli, dalla fase di acquisizione fino alla dismissione, 
garantendo al contempo un equilibrio tra minimizzazione dei costi di mantenimento e massimizzazione dei ricavi derivanti dal noleggio. La natura 
modulare del sistema consente alle imprese di adottare un modello “pay-per-use”, sostenendo esclusivamente i costi relativi alle funzionalità 
effettivamente utilizzate, il che ne amplifica la flessibilità e la scalabilità.

Un aspetto rilevante di 3FS{\textsuperscript{\tiny\textcopyright}} è la gestione automatizzata degli accordi e della rotazione del parco veicoli. Attraverso strumenti di analisi 
predittiva, il sistema è in grado di stimare i prezzi di acquisto dei veicoli, tenendo conto di sconti e condizioni contrattuali, e di 
generare automaticamente la configurazione ottimale della flotta operativa. Questo approccio consente non solo di migliorare il potere negoziale 
con i costruttori automobilistici, ma anche di predisporre processi di rotazione basati su criteri di riduzione dei costi o incremento della 
redditività.

La piattaforma integra inoltre funzionalità avanzate per la distribuzione e messa in servizio dei veicoli, assicurando che le consegne 
avvengano in coerenza con i target mensili della flotta operativa. Ogni fase della logistica, dalla pre-ispezione fino alla distribuzione 
presso le stazioni di noleggio finale, è monitorata e tracciata, riducendo i rischi di non conformità. A livello normativo e assicurativo, 
il sistema automatizza adempimenti quali registrazione delle targhe, pagamento delle tasse di possesso e gestione delle coperture assicurative,
rendendo i veicoli disponibili al noleggio in tempi rapidi.

Durante il ciclo di vita dei veicoli, 3FS{\textsuperscript{\tiny\textcopyright}} gestisce in maniera integrata attività quali manutenzione programmata, rinnovi assicurativi, 
pagamento delle tasse automobilistiche e risoluzione di danni e sinistri. Una caratteristica distintiva è l'interazione diretta con le autorità 
pubbliche per la gestione delle multe, semplificando così procedure tradizionalmente onerose. La fase di de-fleeting, infine, è supportata 
da processi automatizzati di selezione dei veicoli da dismettere, dalla gestione delle aste alla produzione di documentazione fiscale, fino 
alla sincronizzazione con i sistemi contabili aziendali.

Sul piano finanziario e contabile, la piattaforma consente la registrazione e la riconciliazione in tempo reale di fatture di acquisto e vendita, 
nonché la gestione dei flussi di cassa e la produzione di reportistica sia statutaria che gestionale. L'integrazione con sistemi contabili esterni 
amplia ulteriormente le possibilità di controllo e trasparenza, supportando la chiusura mensile e le analisi manageriali. \cite{progesoftware_3fs}

\subsection{Il Ticket Management System applicato a 3FS{\textsuperscript{\tiny\textcopyright}}}
Il Ticket Management System è per ora solo un proof-of-concept, ma in futuro si potrà integrare all'interno di 3FS{\textsuperscript{\tiny\textcopyright}} per fornire supporto tecnico agli utenti.
Obiettivo aggiuntivo è quello di fornire al modello una dettagliata conoscenza del funzionamento del sistema software, così da poter aiutare nella risoluzione
dei problemi più semplici.

L'integrazione con 3FS{\textsuperscript{\tiny\textcopyright}} prevede più livelli funzionali: raccolta dei ticket da interfacce utente (web, mobile e punti di assistenza), correlazione automatica
con moduli del sistema (es. logistica, contabilità, gestione veicoli) e instradamento verso gli operatori o verso risposte automatiche basate su conoscenza
contestualizzata. Il proof-of-concept dimostra la fattibilità della correlazione semantica fra descrizioni dei problemi e componenti software, migliorando 
i tempi di diagnosi e la qualità delle soluzioni proposte.

\subsubsection{Vantaggi attesi}
L'integrazione del Ticket Management System con 3FS{\textsuperscript{\tiny\textcopyright}} mira a ottenere diversi benefici misurabili:
\begin{itemize}
    \item riduzione del tempo medio di risoluzione grazie a suggerimenti contestuali e triage automatico;
    \item diminuzione del carico operativo umano per attività ripetitive e di basso valore;
    \item miglioramento della qualità del servizio e della soddisfazione utente tramite risposte coerenti e tempestive;
    \item raccolta di conoscenza operativa che può alimentare analisi predittive e piani di miglioramento.
\end{itemize}

\section{Conclusione}
Il Model Context Protocol offre un nuovo modo per potenziare le capacità degli LLM, già di per sé strumenti estremamente capaci e potenti.
Questa tesi ha illustrato una possibile applicazione di questa tecnologia.