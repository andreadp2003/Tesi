\setcounter{secnumdepth}{-1}

Il \textit{Ticket Management System} è pensato per essere posizionato all'interno di un contesto più grande, ossia all'interno di un sistema software
per il quale un utente può avere necessità di un supporto tecnico. Il sistema che verrà preso in esame è 3FS{\textsuperscript{\tiny\textcopyright}}.

\section{\texorpdfstring{3FS{\textsuperscript{\tiny\textcopyright}} - Panoramica del funzionamento}{3FS (c) - Panoramica del funzionamento}}
Il sistema 3FS{\textsuperscript{\tiny\textcopyright}} (\textit{Fleet and Financial Flexible System}) rappresenta una soluzione software avanzata per la gestione integrata delle flotte aziendali,
concepita per rispondere in maniera automatizzata e completa a un'ampia gamma di esigenze operative, commerciali, contabili, finanziarie e normative.

L'obiettivo primario della piattaforma è ottimizzare il ciclo di vita dei veicoli, garantendo un equilibrio tra costi e ricavi.
La natura modulare del sistema consente alle imprese di adottare un modello "pay-per-use", sostenendo solo i costi delle funzionalità 
effettivamente utilizzate.

Un aspetto rilevante di 3FS{\textsuperscript{\tiny\textcopyright}} è la gestione automatizzata degli accordi e della rotazione del parco veicoli. Attraverso strumenti di analisi
predittiva, il sistema è in grado di stimare i prezzi di acquisto dei veicoli, tenendo conto di sconti e condizioni contrattuali, e di
generare automaticamente la configurazione ottimale della flotta operativa. Questo approccio consente non solo di migliorare il potere negoziale
con i costruttori automobilistici, ma anche di predisporre processi di rotazione basati su criteri di riduzione dei costi o incremento della
redditività.

La piattaforma integra, inoltre, funzionalità avanzate per la distribuzione e messa in servizio dei veicoli, assicurando che le consegne
avvengano in coerenza con i target mensili della flotta operativa. Ogni fase della logistica, dall'ispezione fino alla distribuzione
presso le stazioni di noleggio finale, è monitorata e tracciata, riducendo i rischi di non conformità. A livello normativo e assicurativo,
il sistema automatizza adempimenti quali registrazione delle targhe, pagamento delle tasse di possesso e gestione delle coperture assicurative,
rendendo i veicoli disponibili al noleggio in tempi rapidi.

Durante il ciclo di vita dei veicoli, 3FS{\textsuperscript{\tiny\textcopyright}} gestisce in maniera integrata attività quali manutenzione programmata, rinnovi assicurativi,
pagamento delle tasse automobilistiche e risoluzione di danni e sinistri. Una caratteristica distintiva è l'interazione diretta con le autorità
pubbliche per la gestione delle multe, semplificando così procedure tradizionalmente onerose. La fase di de-fleeting, infine, è supportata
da processi automatizzati di selezione dei veicoli da dismettere, dalla gestione delle aste alla produzione di documentazione fiscale, fino
alla sincronizzazione con i sistemi contabili aziendali.

Sul piano finanziario e contabile, la piattaforma consente la registrazione e la riconciliazione in tempo reale di fatture di acquisto e vendita,
nonché la gestione dei flussi di cassa e la produzione di reportistica sia statutaria che gestionale. L'integrazione con sistemi contabili esterni
amplia ulteriormente le possibilità di controllo e trasparenza, supportando la chiusura mensile e le analisi manageriali. \cite{progesoftware_3fs}

\newpage
\section{\texorpdfstring{Il Ticket Management System applicato a 3FS{\textsuperscript{\tiny\textcopyright}}}{Il Ticket Management System applicato a 3FS (c)}}
Il Ticket Management System è per ora solo un proof-of-concept, ma in futuro si potrà integrare all'interno di 3FS{\textsuperscript{\tiny\textcopyright}} per fornire supporto tecnico agli utenti.
I ticket saranno dunque inerenti a problemi legati alla piattaforma in questione, il che presuppone che l'LLM abbia una conoscenza dettagliata del suo funzionamento,
anche per poter rispondere in modo autonomo a problematiche semplici.
Questo è un aspetto che non è mai stato affrontato nel progetto così com'è ora: per ora l'unico contesto che il chatbot ha a disposizione sono gli schemi JSON degli oggetti
da creare e non una documentazione estesa.

\newpage
\section{Vantaggi attesi}
L'integrazione del Ticket Management System con 3FS{\textsuperscript{\tiny\textcopyright}} mira a ottenere diversi benefici:
\begin{itemize}
    \item riduzione del tempo medio di risoluzione dei problemi;
    \item diminuzione del carico di lavoro che grava sugli operatori umani;
    \item miglioramento della qualità del servizio e della soddisfazione dell'utente;
\end{itemize}

\newpage
\section{Conclusione}
Il Model Context Protocol offre un nuovo modo per potenziare le capacità degli LLM, già di per sé strumenti estremamente capaci e potenti.
Questa tesi ha illustrato una possibile applicazione di questa tecnologia.
In prospettiva, l'evoluzione di questo protocollo potrà abilitare nuove forme di interazione tra modelli linguistici e sistemi aziendali 
complessi, aprendo la strada a un'integrazione sempre più naturale tra intelligenza artificiale e processi produttivi.