Siamo oramai in un'era in cui l'intelligenza artificiale è presente in ogni aspetto della nostra vita digitale, ancora di più dopo l'uscita di ChatGPT, che infranse il record di applicazione con la più rapida crescita nella storia di Internet, ben 100 milioni di utenti in soli 2 mesi \cite{ubs2023latest}. Gli ultimi anni in particolare sono stati caratterizzati dall’intelligenza artificiale generativa, capace cioè di generare testi, audio, immagini o video a partire da un semplice “prompt” dell’utente, come ad esempio i modelli GPT alla base di ChatGPT stesso. \\
Di particolare interesse sono le IA capaci di generare testo, i cosiddetti Large Language Model (LLM), letteralmente modelli linguistici di grandi dimensioni, in grado di simulare con sorprendente accuratezza le capacità conversazionali di un essere umano. Un'interessante conseguenza del modo in cui questi algoritmi sono stati creati, ma soprattutto la mole di dati usata per addestrarli, è stato l’emergere di comportamenti che potremmo considerare intelligenti, con un livello quasi al pari di quello di un essere umano. Ad esempio, gli LLM sono in grado di ricordare informazioni, generare risposte articolate basate su un ampio contesto e ragionare su problemi complessi derivanti dalle branche più disparate, dalla meccanica quantistica alla biologia evolutiva. \\
Negli anni è così nato il desiderio di rendere ancora più sofisticate le abilità di questi modelli linguistici, finora relegate al “semplice” rispondere ai messaggi degli utenti. Ed è per questo che Anthropic, azienda leader nel settore della ricerca sugli LLM e creatrice dei molto diffusi modelli Claude Sonnet, ha ideato un meccanismo che facilita questo obiettivo: il Model Context Protocol (MCP). Questo protocollo open-source definisce una procedura standard che consente alle applicazioni di fornire contesto ai modelli linguistici, permettendo loro di accedere e interagire con una serie di strumenti esterni in modo sicuro e controllato. In questo modo, i modelli possono eseguire operazioni specifiche, recuperare informazioni o integrare funzionalità aggiuntive, migliorando l’efficacia e la personalizzazione delle risposte generate \cite{modelcontextprotocol2023}. \\
Obiettivo di questa tesi è illustrare un progetto volto a dimostrare nella pratica le funzionalità del Model Context Protocol. Il progetto in questione è un sistema di ticket management, che consente agli utenti di interfacciarsi con un chatbot che rende automatica la creazione dei ticket delineanti i problemi riscontrarti, per poi inoltrarli a degli sviluppatori che possano visionarli ed avviare delle procedure per risolverli.
