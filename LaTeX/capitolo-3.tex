\chapter{Ticket Management System}

Il \textit{Ticket Management System} è un'applicazione software che ha lo scopo di aiutare gli utenti nella creazione e gestione di ticket,
creati per segnalare la presenza di problemi individuati, e nel loro inoltro a degli sviluppatori che possano analizzare e poi risolvere
tali problemi. Funzionalità centrale dell'applicazione è il fatto che l'utente si interfaccia con un chatbot, un'istanza
della risorsa Azure OpenAI che usa il modello GPT-4.1 nello specifico, per usufruire dei servizi offerti.

Il \textit{Ticket Management System} è stato progettato come una \textit{solution} in C\# suddivisa, secondo un approccio modulare,
in più progetti indipendenti ma interconnessi. Tale organizzazione riflette i principi delle moderne architetture software a livelli,
secondo il principio \textit{Separation of Concerns}, favorendo la manutenibilità e la possibilità di estendere il sistema
senza introdurre dipendenze circolari o accoppiamenti eccessivamente rigidi.

La composizione della solution prevede cinque progetti principali, ciascuno con un ruolo ben definito: \textit{TM.Shared}, \textit{TM.Data},
\textit{TM.CQRS}, \textit{TM.Function} e \textit{TM.Client}.

\newpage
\section{Progetti nella solution}

\subsection{TM.Shared}
Il progetto \textit{TM.Shared} definisce tutte le classi degli oggetti che vengono creati internamente nel codice a seguito dell'interazione
con l'utente. Tali classi sono:
\begin{itemize}
    \item \texttt{Ticket}: elemento centrale del sistema; rappresenta i ticket creati dagli utenti, con i relativi \texttt{Comment},
        le \texttt{Task} legate, il \texttt{TicketStatus} e la \texttt{TicketPriority}.
    \item \texttt{Category}: ogni \texttt{Ticket} appartiene ad una categoria.
    \item \texttt{Comment}: i commenti associati ad un determinato \texttt{Ticket}.
    \item \texttt{Task}: i task sono le operazioni affidate agli sviluppatori, generati dal sistema a seguito della creazione
        di un \texttt{Ticket}.
    \item \texttt{User}: gli utenti che usano il sistema; sono intesi anche gli sviluppatori che devono risolvere le \texttt{Task}.
\end{itemize}
Oltre alle classi però, il progetto definisce per ognuna dei DTO (\textit{Data Transfer Objects}) \cite{microsoft_webapi_ef_part5}.
I DTO sono classi compatte che assomigliano alle classi originali ma che contengono meno informazioni, solo quelle ritenute necessarie. 
L'obiettivo è migliorare le performance legate al trasferimento dati tra i diversi sottosistemi, oltre a nascondere alcune logiche interne delle varie classi.

\subsection{TM.Data}
Il progetto \textit{TM.Data} ha il compito di gestire la persistenza delle informazioni, fornendo l'accesso al database
ed archiviando gli oggetti creati. Il database è archiviato all'interno dell'ambiente cloud \textit{Azure} ed utilizza il linguaggio SQL.

\subsection{TM.CQRS}
Il progetto \textit{TM.CQRS} introduce l'implementazione del pattern \textit{Command Query Responsibility Segregation} \cite{azure_cqrs_msdocs}.
Il CQRS è un pattern architetturale che ha lo scopo di isolare le operazioni di lettura (\texttt{Queries}) e di scrittura (\texttt{Commands})
in un progetto separato, che possono poi essere richiamate da ognuno degli altri progetti per effetturare query sul database.

\subsection{TM.Function}
Il cuore della logica applicativa è rappresentato da \textit{TM.Function}. Essa è un'\textit{Azure Function}, ovvero una soluzione serverless
che consente agli sviluppatori di scrivere ed eseguire codice sul cloud in maniera più efficace e compatta, senza dover ricorrere alla 
creazione di un progetto \cite{azurefunctions_msdocs}. Prende il ruolo dell'MCP Host, l'applicazione IA che gestisce la logica principale del
Model Context Protocol. Verrà discusso più approfonditamente in seguito.

\subsection{TM.Client}
Infine, \textit{TM.Client} costituisce l'interfaccia utente, nonchè l'MCP Client. Esso contiene la sola classe \texttt{Program.cs}, occupandosi
di avviare l'intera solution ed integrare i vari altri progetti, oltre che inizializzare la comunicazione con il chatbot tramite terminale.

\newpage
\section{Il Model Context Protocol in pratica}
Nucleo centrale del progetto è la classe \texttt{TicketManagementTools} del progetto TM.Function. Il suo obiettivo è quello di definire una serie
di \textit{tools} che l'LLM invoca per eseguire le operazioni richieste. Vediamo un esempio:

\begin{scriptsize}
\begin{verbatim}
[Function("AggiungiTicket")] 
public async Task<TicketDtoCreate?> AggiungiTicketAsync(
    [McpToolTrigger("AggiungiTicket", "Aggiunge un ticket")] ToolInvocationContext context,
    [McpToolProperty("Ticket", "string", DtoJsonSchemas.TicketDtoCreateSchema)] string ticketDto)
{
    var options = new JsonSerializerOptions
    {
        PropertyNameCaseInsensitive = true,
        Converters = { new JsonStringEnumConverter() }
    };

    var dto = JsonSerializer.Deserialize<TicketDtoCreate>(ticketDto, options);
    if (dto == null || !dto.IsValid())
    {
        return null;
    }

    var entity = dto.ToEntity();
    await WriteRepo<Ticket>().AddAsync(entity);
    await WriteRepo<Ticket>().SaveChangesAsync();

    return dto;
}
\end{verbatim}
\end{scriptsize}

Questo metodo, come si può facilmente dedurre, permette di creare un oggetto \texttt{Ticket} (più precisamente un DTO \texttt{TicketDtoCreate}).
Caratteristica del Model Context Protocol sono le sue decorazioni: \texttt{McpToolProperty} e \texttt{McpToolTrigger}. Curiosamente, queste
decorazioni esplicitano i propri dettagli operativi in linguaggio naturale e non in un linguaggio di programmazione, C\# in questo caso.
Nonostante ciò, l'LLM è in grado di interpretare letteralmente queste istruzioni ed utilizzare correttamente i tool che gli sono posti a
disposizione.

Da notare l'uso di JSON all'interno del metodo. Per comunicare al chatbot la struttura degli oggetti da creare, gli vengono passati
degli schemi JSON ai quali l'LLM dev'essere conforme. Nel frammento mostrato, lo schema è fornito tramite \texttt{DtoJsonSchemas.TicketDtoCreateSchema}
nell'attributo \texttt{McpToolProperty}; questo schema definisce la forma del JSON atteso e permette di vincolare l'output dell'LLM alla struttura prevista.

Il parametro \texttt{ToolInvocationContext} nell'esempio fornisce, come si vede dalla firma, il contesto dell'invocazione dello tool:
in pratica rappresenta il punto di integrazione tra la chiamata esterna (o generata dall'LLM) e l'implementazione della Function stessa.
La presenza di tale contesto e delle specifiche in linguaggio naturale nelle decorazioni facilita la correlazione tra le istruzioni fornite al modello
e le operazioni concrete eseguite dall'host serverless.

Questi elementi insieme mostrano come il Model Context Protocol consenta di esporre funzioni applicative al modello in modo strutturato:
le descrizioni in linguaggio naturale rendono comprensibili i comportamenti dei tool all'LLM, mentre gli schemi JSON e la logica di validazione
garantiscono che i dati prodotti rispettino le aspettative del dominio applicativo.