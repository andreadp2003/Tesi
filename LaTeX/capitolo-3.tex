\chapter{Ticket Management System}

Il \textit{Ticket Management System} è un sistema software che ha lo scopo di aiutare gli utenti nella creazione e gestione di ticket, 
creati per segnalare la presenza di problemi individuati, e nel loro inoltro a degli sviluppatori che possano analizzare e poi risolvere 
tali problemi. Funzionalità centrale è il fatto che l'utente si interfaccia con un chatbot, nello specifico, un'istanza della risorsa 
Azure OpenAI che usa il modello GPT-4.1, per usufruire dei servizi offerti del sistema.
\\
Il \textit{Ticket Management System} è stato progettato come una \textit{solution} in C\# suddivisa, secondo un approccio modulare,
in più progetti indipendenti ma interconnessi. Tale organizzazione riflette i principi delle moderne architetture software a livelli, 
secondo il principio \textit{Separation of Concerns}, favorendo la manutenibilità e la possibilità di estendere il sistema
senza introdurre dipendenze circolari o accoppiamenti eccessivamente rigidi.
\\
La composizione della solution prevede cinque progetti principali, ciascuno con un ruolo ben definito: \textit{TM.Shared}, \textit{TM.Data},
\textit{TM.CQRS}, \textit{TM.Function} e \textit{TM.Client}.
\\
Il progetto \textit{TM.Shared} definisce tutte le classi degli oggetti che vengono creati internamente nel codice al momento dell'interazione 
con l'utente. Tali classi sono:
\begin{itemize}
    \item \texttt{Ticket}: elemento centrale del sistema; rappresenta i ticket creati dagli utenti, con i relativi \texttt{Comment}, 
        le \texttt{Task} legate, il \texttt{TicketStatus} e la \texttt{TicketPriority}.
    \item \texttt{Category}: ogni \texttt{Ticket} appartiene ad una categoria.
    \item \texttt{Comment}: i \texttt{Ticket} hanno una serie di commenti.
    \item \texttt{Task}: i task sono le operazioni affidate agli sviluppatori, generati dal sistema a seguito della creazione 
        di un \texttt{Ticket}. 
    \item \texttt{User}: gli utenti che usano il sistema; sono intesi anche gli sviluppatori che devono risolvere le \texttt{Task}.
\end{itemize}
\\
Il progetto \textit{TM.Data} ha il compito di gestire la persistenza delle informazioni, fornendo l'accesso al database 
ed archiviando gli oggetti creati.
\\
Il progetto \textit{TM.CQRS} introduce l'implementazione del pattern \textit{Command Query Responsibility Segregation}, un pattern 
architetturale che separa le operazioni di lettura (\texttt{Queries}) da quelle di scrittura (\texttt{Commands}) all'interno di un'applicazione. 
\\
Il cuore della logica applicativa è rappresentato da \textit{TM.Function}. Essa è un'Azure Function, ovvero una soluzione serverless 
che consente di scrivere meno codice, gestire un'infrastruttura meno complessa e risparmiare sui costi. Non è più necessario preoccuparsi 
della distribuzione e della gestione dei server, in quanto l'infrastruttura cloud fornisce tutte le risorse aggiornate necessarie per mantenere 
le applicazioni in esecuzione \cite{azurefunctions_msdocs}. Prende il ruolo dell'MCP Host, l'applicazione IA che gestisce la logica principale del
Model Context Protocol. Verrà discusso più approfonditamente in seguito. 
\\
Infine, \textit{TM.Client} costituisce l'interfaccia utente, nonchè l'MCP Client. Esso contiene la sola classe \texttt{Program.cs}, occupandosi
di avviare l'intera solution e integrare i vari altri progetti, oltre che inizializzare la comunicazione con il chatbot.

\section{Il Model Context Protocol in pratica}

